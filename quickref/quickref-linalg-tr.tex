% Sage Linear Algebra Quick Reference
% (c) 2009 by Robert A. Beezer
% Licensed with the GNU Free Documentation License (GFDL)
%   http://www.gnu.org/copyleft/fdl.html
%
%  History
%
%    2009-05-10  Initial version based on Sage 3.4
%    2009-05-13  Added right paren to vector u in "vector operations" section
%
%
\documentclass{article}
\usepackage{graphicx}  
\usepackage[landscape]{geometry}
\usepackage[pdftex]{color}
\usepackage{url}
\usepackage{multicol}
\usepackage{amsmath}
\usepackage{amsfonts}
\newcommand{\ex}{\color{blue}}
\newcommand{\warn}{\bf\color{red}}
\pagestyle{empty}
\advance\topmargin-.9in
\advance\textheight2in
\advance\textwidth3.0in
\advance\oddsidemargin-1.45in
\advance\evensidemargin-1.45in
\parindent0pt
\parskip2pt
% Section break, dictates column widths?
\newcommand{\hr}{\centerline{\rule{3.5in}{1pt}}}
% Adjust gap to affect spacing, page count
\newcommand{\sect}[1]{\hr\par\vspace*{2pt}\textbf{#1}\par}
% Mandatory indentation on subsidiary lines
\newcommand{\skipin}{\hspace*{12pt}}


      \usepackage[utf8]{inputenc}

      \usepackage[T1]{fontenc}
      
\begin{document}
\begin{multicols*}{3}
\begin{center}
%\textbf{Sage Quick Reference: Linear Algebra}\\
\textbf{Kısa Sage Kılavuzu: Doğrusal Cebir}\\
Robert A. Beezer (Türkçeleştiren Kürşat Aker)\\
%Sage Version 3.4
Sage Sürüm 3.4\\
\url{http://wiki.sagemath.org/quickref}\\
%GNU Free Document License, extend for your own use\\
GNU Özgür Belge Lisansı, Dileğinize göre geliştirin\\
%Based on work by Peter Jipsen, William Stein 
P. Jipsen ve W. Stein'ın çalışmalarını esas alarak
\end{center}
% backup over center environment gap
\vspace{-4ex}
%*********************************************
%\sect{Vector Constructions}
\sect{Vektörlerin Tanımlanması}
%{\warn Caution}: First entry of a vector is numbered 0\\
{\warn Uyarı}: Bir vektörün ilk indisi $0$'dır.\\
%{\ex\verb!u = vector(QQ, [1, 3/2, -1])!} length 3 over rationals\\
{\ex\verb!u = vector(QQ, [1, 3/2, -1])!} $= (1,\frac{3}{2},-1) \in \mathbb{Q}^3$ \\
{\ex\verb!v = vector(QQ, {2:4, 95:4, 210:0})!}\\
%\skipin 211 entries, nonzero in entry 4 and entry 95, sparse
\skipin toplam 211 kutu, 2. ve 95. kutular = 4, seyrek 
\par
%*********************************************
%\sect{Vector Operations}
\sect{Vektör İşlemleri}
{\ex\verb!u = vector(QQ, [1, 3/2, -1])!}\\
{\ex\verb!v = vector(ZZ, [1, 8, -2])!}\\
%{\ex\verb!2*u - 3*v!} linear combination\\
{\ex\verb!2*u - 3*v!} doğrusal bileşim\\
{\ex\verb!u.dot_product(v)!} \\
{\ex\verb!u.dot_product(v)!} iç çarpım\\
%{\ex\verb!u.cross_product(v)!}  order: \verb!u!$\times$\verb!v!\\
{\ex\verb!u.cross_product(v)!}  sıralama: \verb!u!$\times$\verb!v!\\
%{\ex\verb!u.inner_product(v)!}  inner product matrix from parent\\
{\ex\verb!u.inner_product(v)!}  u'nun anasından gelen iç çarpım\\
%{\ex\verb!u.pairwise_product(v)!} vector as a result\\
{\ex\verb!u.pairwise_product(v)!} $=(u_0v_0, u_1v_1, u_2v_2, \ldots)$ \\
%{\ex\verb!u.norm() == u.norm(2)!} Euclidean norm\\
{\ex\verb!u.norm() == u.norm(2)!} Öklid normu\\
%{\ex\verb!u.norm(1)!} sum of entries\\
{\ex\verb!u.norm(1)!} kutu içeriklerinin toplamı\\
%{\ex\verb!u.norm(Infinity)!} maximum entry\\
{\ex\verb!u.norm(Infinity)!} en büyük kutu içeriği\\
%{\ex\verb!A.gram_schmidt()!} converts the rows of matrix \verb!A!
{\ex\verb!A.gram_schmidt()!} \verb!A! matrisinin satırlarını dönüştürür
\par
%*********************************************
%\sect{Matrix Constructions}
\sect{Matrislerin Tanımlanması}
%{\warn Caution}: Row, column numbering begins at 0\\
{\warn Uyarı}: Satır ve sütunların isimlendirilmesi 0'dan başlar\\
{\ex\verb!A = matrix(ZZ, [[1,2],[3,4],[5,6]])!}\\
%\skipin $3\times 2$ over the integers\\
\skipin Tamsayılar üzerinde $3\times 2$-lik matris\\
{\ex\verb!B = matrix(QQ, 2, [1,2,3,4,5,6])!}\\
%\skipin 2 rows from a list, so $2\times 3$ over rationals\\
\skipin Listeden 2-şer satır al, kesirlerde $2\times 3$-lük  matris yap\\
{\ex\verb!C = matrix(CDF, 2, 2, [[5*I, 4*I], [I, 6]])!}\\
%\skipin complex entries, 53-bit precision\\
\skipin kutuları karmaşık sayılar, 53-bit hassasiyet\\
%{\ex\verb!Z = matrix(QQ, 2, 2, 0)!} zero matrix\\
{\ex\verb!Z = matrix(QQ, 2, 2, 0)!} sıfır matris\\
{\ex\verb!D = matrix(QQ, 2, 2, 8)!}\\
%\skipin diagonal entries all $8$, other entries zero\\
\skipin köşegendeki terimler $8$, geri kalanlar $0$\\
%{\ex\verb!I = identity_matrix(5)!} $5\times 5$ identity matrix\\
{\ex\verb!I = identity_matrix(5)!} $5\times 5$-lik birim matris\\
{\ex\verb!J = jordan_block(-2,3)!}\\
%\skipin $3\times 3$ matrix, $-2$ on diagonal, $1$'s on super-diagonal
\skipin $3\times 3$-lük matris, köşegende $-2$, köşegenüstünde $1$'ler
{\ex\verb!var('x y z'); K = matrix(SR, [[x,y+z],[0,x^2*z]])!}\\
%\skipin symbolic expressions live in the ring \verb!SR!\\
\skipin Simgesel ifadeler, \verb!SR! halkasında yaşarlar\\
{\ex\verb!L=matrix(ZZ, 20, 80, {(5,9):30, (15,77):-6})!}\\
%\skipin $20\times 80$, two non-zero entries, sparse representation
\skipin $20\times 80$-lik matris, iki tane $\neq 0$ kutu, seyrek temsil
\par
%*********************************************
%\sect{Matrix Multiplication}
\sect{Matris Çarpımı}
{\ex\verb!u = vector(QQ, [1,2,3])!},\quad{\ex\verb!v = vector(QQ, [1,2])!}\\
{\ex\verb!A = matrix(QQ, [[1,2,3],[4,5,6]])!}\\
{\ex\verb!B = matrix(QQ, [[1,2],[3,4]])!}\\
{\ex\verb!u*A!},\quad
{\ex\verb!A*v!},\quad
{\ex\verb!B*A!},\quad
{\ex\verb!B^6!},\quad
%{\ex\verb!B^(-3)!} all possible\\
{\ex\verb!B^(-3)!}'nün tümü geçerlidir.\\
%{\ex\verb!B.iterates(v, 6)!}   produces $vB^0, vB^1,\dots, vB^5$\\
{\ex\verb!B.iterates(v, 6)!}   $vB^0, vB^1,\dots, vB^5$'u üretir\\
\skipin\verb!rows = False! \verb!v! vektörünü matrislerin sağına koyar\\
%{\ex\verb!f(x)=x^2+5*x+3!} then {\ex\verb!f(B)!} is possible\\
{\ex\verb!f(x)=x^2+5*x+3!} tanımlıysa, {\ex\verb!f(B)!} geçerlidir\\
%{\ex\verb!B.exp()!} matrix exponential, i.e.\ $\sum_{k=0}^{\infty}\frac{B^k}{k!}$
{\ex\verb!B.exp()!} $= e^B = \sum_{k=0}^{\infty}\frac{B^k}{k!}$
\par
%*********************************************
%\sect{Matrix Spaces}
\sect{Matris Uzayları}
{\ex\verb!M = MatrixSpace(QQ, 3, 4)!}\\
%\skipin dimension 12 space of $3\times 4$ matrices\\
\skipin $3\times 4$-lük matrislerin 12 boyutlu uzayı\\
{\ex\verb!A = M([1,2,3,4,5,6,7,8,9,10,11,12])!}\\
%\skipin is a $3\times 4$ matrix, an element of \verb!M!\\
\skipin $3\times 4$-lük bir matris, \verb!M! uzayının bir üyesidir\\
%{\ex\verb!M.basis()!} \\
%{\ex\verb!M.dimension()!}\\
%{\ex\verb!M.zero_matrix()!}
{\ex\verb!M.basis()!} \verb!M!'nin tabanı \\
{\ex\verb!M.dimension()!} \verb!M!'nin boyutu\\
{\ex\verb!M.zero_matrix()!} \verb!M!'nin sıfır matrisi
\par
%*********************************************
%\sect{Matrix Operations}
\sect{Matris İşlemleri}
%{\ex\verb!5*A+2*B!} linear combination\\
{\ex\verb!5*A+2*B!} doğrusal bileşim\\
%{\ex\verb!A.inverse()!}, also {\ex\verb!A^(-1)!}, {\ex\verb!~A!}\\
{\ex\verb!A.inverse()!}, {\ex\verb!A^(-1)!}, {\ex\verb!~A!} = \verb!A!'nın tersi\\
%\skipin\verb!ZeroDivisionError! if singular\\
\skipin Eğer \verb!A! tersinir değilse, \verb!ZeroDivisionError! hatası gelir\\
%{\ex\verb!A.transpose()!}\\
{\ex\verb!A.transpose()!} $=\verb!A!^t=$ \verb!A!'nın devriği \\
%{\ex\verb!A.antitranspose()!} transpose + reverse orderings\\
{\ex\verb!A.antitranspose()!} $\verb!A!^t$'nin satır ve sütunları ters sıralı\\
%{\ex\verb!A.adjoint()!}  matrix of cofactors\\
{\ex\verb!A.adjoint()!}  Kofaktörler matrisi\\
%{\ex\verb!A.conjugate()!} entry-by-entry complex conjugates\\
{\ex\verb!A.conjugate()!} kutu kutu karmaşık eşlenikler\\
%{\ex\verb!A.restrict(V)!} restriction on invariant subspace \verb!V!
{\ex\verb!A.restrict(V)!} \verb!A!'nın korunan \verb!V! altuzayına kısıtlaması
\par
%*********************************************
%\sect{Row Operations}
\sect{Satır/Sütun İşlemleri}
%Row Operations: (change matrix in place)\\
Satır İşlemleri: (matrisi değiştirirler)\\
%{\warn Caution}: first row is numbered 0\\
{\warn Uyarı}: İlk satırın indisi 0'dır\\
%{\ex\verb!A.rescale_row(i,a)!} \verb!a*!(row \verb!i!)\\
{\ex\verb!A.rescale_row(i,a)!} \verb!a*!(\verb!i! satırı)\\
%{\ex\verb!A.add_multiple_of_row(i,j,a)!} \verb!a*!(row \verb!j!) + row \verb!i!\\
{\ex\verb!A.add_multiple_of_row(i,j,a)!} \verb!a*!(\verb!j! satırı) $+$ \verb!i! satırı\\
%{\ex\verb!A.swap_rows(i,j)!} \\
{\ex\verb!A.swap_rows(i,j)!}  \verb!i! satırıyla \verb!j! satırı yer değiştirir \\
%Each has a column variant, \verb!row!$\rightarrow$\verb!col!\\
Her satır işleminin sütun karşılığı vardır, \verb!row!$\rightarrow$\verb!col!\\
%For a new matrix, use e.g.\ {\ex\verb!B = A.with_rescaled_row(i,a)!}
Yeni bir matris için, örn. \ {\ex\verb!B = A.with_rescaled_row(i,a)!}
\par
%*********************************************
%\sect{Echelon Form}
\sect{Matrisleri Merdiven Haline Getirilmesi }
{\ex\verb!A.echelon_form()!},
{\ex\verb!A.echelonize()!},
{\ex\verb!A.hermite_form()!}\\
%{\warn Caution}: Base ring affects results\\
{\warn Uyarı}: Sonuçlar, tanım halkasına bağlıdır \\
{\ex\verb!A = matrix(ZZ,[[4,2,1],[6,3,2]])!}\\
{\ex\verb!B = matrix(QQ,[[4,2,1],[6,3,2]])!}\\
\begin{tabular}{ll}
{\ex\verb!A.echelon_form()!} & {\ex\verb!B.echelon_form()!}\\
$\left(\begin{array}{rrr}
2 & 1 & 0 \\
0 & 0 & 1
\end{array}\right)$
&
$\left(\begin{array}{rrr}
1 & \frac{1}{2} & 0 \\
0 & 0 & 1
\end{array}\right)$
\end{tabular}\\
%{\ex\verb!A.pivots()!} indices of columns spanning column space\\
%{\ex\verb!A.pivot_rows()!} indices of rows spanning row space
{\ex\verb!A.pivots()!} sütun uzayını geren sütunların indisleri\\
{\ex\verb!A.pivot_rows()!} satır uzayını geren satırların indisleri
\par
%*********************************************
%\sect{Pieces of Matrices}
\sect{Matrislerin Parçaları}
%{\warn Caution}: row, column numbering begins at 0\\
{\warn Uyarı}: Satır ve sütunların isimlendirilmesi 0'dan başlar\\
{\ex\verb!A.nrows()!}\\
{\ex\verb!A.ncols()!}\\
%{\ex\verb!A[i,j]!} entry in row \verb!i! and column \verb!j!\\
{\ex\verb!A[i,j]!} \verb!i!-nci satır ve \verb!j!-nci sütundaki kutunun içeriği\\
%\skipin{\warn Caution}: OK: \verb!A[2,3] = 8!, Error: \verb!A[2][3] = 8!\\
\skipin{\warn Uyarı}: Doğru: \verb!A[2,3] = 8!, Yanlış: \verb!A[2][3] = 8!\\
%{\ex\verb!A[i]!} row \verb!i! as immutable Python tuple\\
{\ex\verb!A[i]!} \verb!i! satırı (değiştirilemez bir Python destesi olarak)\\
%{\ex\verb!A.row(i)!} returns row \verb!i! as Sage vector\\
{\ex\verb!A.row(i)!} \verb!i! satırı (Sage vektörü)\\
%{\ex\verb!A.column(j)!} returns column \verb!j! as Sage vector\\
{\ex\verb!A.column(j)!} \verb!j! sütunu (Sage vektörü)\\
%{\ex\verb!A.list()!} returns single Python list, row-major order\\
{\ex\verb!A.list()!} Matris, satır öncelikli dizilmiş (Python listesi)\\
{\ex\verb!A.matrix_from_columns([8,2,8])!}\\
%\skipin new matrix from columns in list, repeats OK\\
\skipin seçilen sütunlardan matris yap (sütunlar yinelenebilir)\\
{\ex\verb!A.matrix_from_rows([2,5,1])!}\\
%\skipin new matrix from rows in list, out-of-order OK\\
\skipin seçilen satırlardan matris yap (satırlar sırasız olabilir)\\
{\ex\verb!A.matrix_from_rows_and_columns([2,4,2],[3,1])!}\\
%\skipin common to the rows and the columns\\
\skipin satır ve sütun numaralı kutulardan yeni matris yap\\
%{\ex\verb!A.rows()!}  all rows as a list of tuples\\
{\ex\verb!A.rows()!}  tüm satırlar (destelerden bir liste halinde)\\
%{\ex\verb!A.columns()!}  all columns as a list of tuples\\
{\ex\verb!A.columns()!}  tüm sütunlar (destelerden bir liste halinde)\\
{\ex\verb!A.submatrix(i,j,nr,nc)!}\\
%\skipin start at entry \verb!(i,j)!, use \verb!nr! rows, \verb!nc! cols\\
\skipin sol üst köşesi \verb!(i,j)! kutusu olan \verb!nr! $\times$ \verb!nc! altmatris\\
%{\ex\verb!A[2:4,1:7]!}, {\ex\verb!A[0:8:2,3::-1]!} Python-style list slicing
{\ex\verb!A[2:4,1:7]!}, {\ex\verb!A[0:8:2,3::-1]!} Python tarzı liste kesiti
\par
%*********************************************
%\sect{Combining Matrices}
\sect{Matrisleri Birleştirme}
%{\ex\verb!A.augment(B)!}  \verb!A! in first columns, \verb!B! to the right\\
{\ex\verb!A.augment(B)!}  $=[\verb!A|B!]$, \verb!A! matrisi \verb!B!'nin soluna konur\\
%{\ex\verb!A.stack(B)!}  \verb!A! in top rows, \verb!B! below\\
{\ex\verb!A.stack(B)!}  \verb!A! matrisi \verb!B!'nin üzerine konur\\
%{\ex\verb!A.block_sum(B)!}  Diagonal, \verb!A! upper left, \verb!B! lower right\\
{\ex\verb!A.block_sum(B)!} \verb!A! solüstte, \verb!B! sağaltta blok köşegen mat.\\
%{\ex\verb!A.tensor_product(B)!}  Multiples of \verb!B!, arranged as in \verb!A!
{\ex\verb!A.tensor_product(B)!} \verb!B!'nin katları \verb!A!'ya uygun dizilmiş
\par
%*********************************************
%\sect{Scalar Functions on Matrices}
\sect{Matrislere Sayı Atayan Fonksiyonlar}
{\ex\verb!A.rank()!}\\
{\ex\verb!A.nullity() == A.left_nullity()!}\\
{\ex\verb!A.right_nullity()!}\\
{\ex\verb!A.determinant() == A.det()!}\\
{\ex\verb!A.permanent()!}\\
{\ex\verb!A.trace()!}\\
%{\ex\verb!A.norm() == A.norm(2)!} Euclidean norm\\
%{\ex\verb!A.norm(1)!} largest column sum\\
%{\ex\verb!A.norm(Infinity)!} largest row sum\\
%{\ex\verb!A.norm('frob')!} Frobenius norm
{\ex\verb!A.norm() == A.norm(2)!} Öklid normu\\
{\ex\verb!A.norm(1)!} en büyük sütun toplamı\\
{\ex\verb!A.norm(Infinity)!} en büyük satır toplamı\\
{\ex\verb!A.norm('frob')!} Frobenius normu
\par
%*********************************************
%\sect{MatrixProperties}
\sect{Matrislerin Özellikleri}
%{\ex\verb!.is_zero()!} (totally?), 
%{\ex\verb!.is_one()!} (identity matrix?),\\
%{\ex\verb!.is_scalar()!} (multiple of identity?), 
%{\ex\verb!.is_square()!},\\
%{\ex\verb!.is_symmetric()!}, 
%{\ex\verb!.is_invertible()!}, 
%{\ex\verb!.is_nilpotent()!}
{\ex\verb!.is_zero()!} (sıfır mı?), 
{\ex\verb!.is_one()!} (birim matris mi?),\\
{\ex\verb!.is_scalar()!} (birimin katı mı?), 
{\ex\verb!.is_square()!},\\
{\ex\verb!.is_symmetric()!}, 
{\ex\verb!.is_invertible()!}, 
{\ex\verb!.is_nilpotent()!}
\par
%*********************************************
%\sect{Eigenvalues}
\sect{Özdeğerler}
%{\ex\verb!A.charpoly('t')!}  no variable specified defaults to \verb!x!\\
{\ex\verb!A.charpoly('t')!} $=\det(\verb!t! - \verb!A!)$ (varsayılan değişken \verb!x!)\\
{\ex\verb!  A.characteristic_polynomial() == A.charpoly()!}\\
%{\ex\verb!A.fcp('t')!}  factored characteristic polynomial\\
{\ex\verb!A.fcp('t')!} çarpanlarına ayrılmış karakteristik polinom\\
%{\ex\verb!A.minpoly()!}  the minimum polynomial\\
{\ex\verb!A.minpoly()!}  minimum polinom\\
{\ex\verb!  A.minimal_polynomial() == A.minpoly()!}\\
%{\ex\verb!A.eigenvalues()!} unsorted list, with mutiplicities\\
{\ex\verb!A.eigenvalues()!} Özdeğerlerin listesi (sırasız, katlılık) \\
%
%{\ex\verb!A.eigenvectors_left()!} vectors on left, \verb!_right! too\\
%\skipin Returns a list of triples, one per eigenvalue:\\
%\skipin\skipin\verb!e!: the eigenvalue\\
%\skipin\skipin\verb!V!: list of vectors, basis for eigenspace\\
%\skipin\skipin\verb!n!: algebraic multiplicity\\
%
{\ex\verb!A.eigenvectors_left()!} vektörler solda (sağ $\to $\verb!_right!) \\
\skipin Özdeğerlere karşılık üçlülerden $\verb!(e,V,n)!$ bir liste verir:\\
\skipin\skipin\verb!e!: Özdeğer\\
\skipin\skipin\verb!V!: Özuzayın bir tabanı\\
\skipin\skipin\verb!n!: Cebirsel Katlılık\\
%%
%{\ex\verb!A.eigenmatrix_right()!} vectors on right, \verb!_left! too\\
%\skipin Returns two matrices:\\
%\skipin\skipin\verb!D!: diagonal matrix with eigenvalues\\
%\skipin\skipin\verb!P!: eigenvectors as columns (rows for left version)\\
%\skipin\skipin\skipin has zero columns if matrix not diagonalizable
%
{\ex\verb!A.eigenmatrix_right()!} vektörler sağda \verb!_left! too\\
\skipin İki matris verir:\\
\skipin\skipin\verb!D!: özdeğerlerden oluşan köşegen matris\\
\skipin\skipin\verb!P!: sütunları \verb!A!'nın özvektörleri (\verb!_left! satırlar özvek.)\\
\skipin\skipin\skipin eğer \verb!A! köşegenlenemezse, sıfır sütunları var
\par                    
%*********************************************
\sect{Matris Ayrışmaları}
%
%{\warn Note}: availability depends on base ring of matrix\\
{\warn Önemli}: Bu yöntemler, uygun halkalarda tanımlıdır.\\
%{\ex\verb!A.jordan_form(transformation=True)!}\\
%\skipin returns a pair of matrices:\\
%\skipin\skipin\verb!J!: matrix of Jordan blocks for eigenvalues\\
%\skipin\skipin\verb!P!: nonsingular matrix\\
%\skipin\skipin so \verb!    A == P^(-1)*J*P!\\
%%
{\ex\verb!A.jordan_form(transformation=True)!}\\
\skipin iki matris verir:\\
\skipin\skipin\verb!J!: Jordan blokların oluşan bir matris\\
\skipin\skipin\verb!P!: tersinir bir matris\\
\skipin\skipin öyle ki, \verb!    A == P^(-1)*J*P!\\
%
%{\ex\verb!A.smith_form()!} returns a triple of matrices:\\
%\skipin\verb!D!: elementary divisors on diagonal\\
%\skipin\verb!U, V!: with unit determinant\\
%\skipin so \verb!    D == U*A*V!\\
%%
{\ex\verb!A.smith_form()!} üç matris verir:\\
\skipin\verb!D!: köşegen boyunca temel bölenler\\
\skipin\verb!U, V!: determinantları tersinir\\
\skipin öyle ki, \verb!    D == U*A*V!\\
%
%{\ex\verb!A.LU()!} returns a triple of matrices:\\
%\skipin\verb!P!: a permutation matrix\\
%\skipin\verb!L!: lower triangular matrix\\
%\skipin\verb!U!: upper triangular matrix\\
%\skipin so \verb!    P*A == L*U!\\
%%
{\ex\verb!A.LU()!} üç  matris verir:\\
\skipin\verb!P!: bir permütasyon matrisi\\
\skipin\verb!L!: alt-üçgen matris\\
\skipin\verb!U!: üst-üçgen matris\\
\skipin öyle ki, \verb!    P*A == L*U!\\
%
%{\ex\verb!A.QR()!} returns a pair of matrices:\\
%\skipin\verb!Q!: an orthogonal matrix\\
%\skipin\verb!R!: upper triangular matrix\\
%\skipin so \verb!    A == Q*R!\\
%%
{\ex\verb!A.QR()!} iki matris verir:\\
\skipin\verb!Q!: dik matris\\
\skipin\verb!R!: üst-üçgen matris\\
\skipin öyle ki, \verb!    A == Q*R!\\
%
%{\ex\verb!A.SVD()!} returns a triple of matrices:\\
%\skipin\verb!U!: an orthogonal matrix\\
%\skipin\verb!S!: zero off the diagonal, same dimensions as \verb!A!\\
%\skipin\verb!V!: an orthogonal matrix\\
%\skipin so \verb!  A == U*S*(V-conjugate-transpose)!\\
%%
{\ex\verb!A.SVD()!} üç matris verir:\\
\skipin\verb!U!: dik matris\\
\skipin\verb!S!: köşegen dışında sıfır, \verb!A! ile aynı boyutlarda\\
\skipin\verb!V!: dik matris\\
\skipin öyle ki, \verb!  A == U*S*(V'nin devrik eşleniği)!\\
%
{\ex\verb!A.symplectic_form()!}\\
{\ex\verb!A.hessenberg_form()!}\\
{\ex\verb!A.cholesky()!}\par
%*********************************************
%\sect{Solutions to Systems}
%{\ex\verb!A.solve_right(B)!} \verb!_left! too\\
%\skipin is solution to \verb!A*X = B!, where \verb!X! is a vector {\bf or} matrix\\
%{\ex\verb!A = matrix(QQ, [[1,2],[3,4]])!}\\
%{\ex\verb!b = vector(QQ, [3,4])!}\\
%\skipin then {\ex\verb!A\b!} returns the solution {\ex\verb!(-2, 5/2)!}
\sect{Doğrusal Denklem Takımlarının Çözümleri}
{\ex\verb!A.solve_right(B)!}, {\ex\verb!A.solve_left(B)!} sırasıyla \verb!A*X = B! 
\skipin ve \verb!X*A = B! denklemlerini çözer (\verb!X!  vektör veya matris)\\
{\ex\verb!A = matrix(QQ, [[1,2],[3,4]])!}\\
{\ex\verb!b = vector(QQ, [3,4])!}\\
\skipin için  \verb!A*x = b!'nin çözümü {\ex\verb!A\b!}$=\verb!A!^{-1}\verb!b!=${\ex\verb!(-2, 5/2)!}'tir.
\par
%*********************************************
%\sect{Vector Spaces}
%{\ex\verb!U = VectorSpace(QQ, 4)!}  dimension 4, rationals as field\\
%{\ex\verb!V = VectorSpace(RR, 4)!}  ``field'' is 53-bit precision reals\\
%{\ex\verb!W = VectorSpace(RealField(200), 4)!}\\
%\skipin ``field'' has 200 bit precision\\
%{\ex\verb!X = CC^4!} 4-dimensional, 53-bit precision complexes\\
%{\ex\verb!Y = VectorSpace(GF(7), 4)!}  finite\\
%\skipin{\ex\verb!Y.finite()!} returns \verb!True!\\
%\skipin{\ex\verb!len(Y.list())!} returns $7^4=2401$ elements
\sect{Doğrusal Uzaylar}
{\ex\verb!U = VectorSpace(QQ, 4)!} kesirler üstünde 4 boyutlu uzay\\
{\ex\verb!V = VectorSpace(RR, 4)!}  ``cisim'' 53-bit hassas gerçeller\\
{\ex\verb!W = VectorSpace(RealField(200), 4)!}\\
\skipin ``cisim'' 200-bit hassaslıkta\\
{\ex\verb!X = CC^4!} 4-boyutlu, 53-bit hassas karmaşıklar\\
{\ex\verb!Y = VectorSpace(GF(7), 4)!}  sonlu doğrusal uzay\\
\skipin{\ex\verb!Y.finite()!} sorusu Doğru (\verb!True!) yanıtı verir\\
\skipin{\ex\verb!len(Y.list())!} $=7^4=2401=$  \verb!Y!'nin üye sayısı 
\par
%*********************************************
%\sect{Vector Space Properties}
\sect{Doğrusal Uzayların Özellikleri}
%{\ex\verb!V.dimension()!}\\
%{\ex\verb!V.basis()!}\\
%{\ex\verb!V.echelonized_basis()!}\\
%{\ex\verb!V.has_user_basis()!} with non-canonical basis?\\
%{\ex\verb!V.is_subspace(W)!} \verb!True! if \verb!W! is a subspace of \verb!V!\\
%{\ex\verb!V.is_full()!} rank equals degree (as module)?\\
%{\ex\verb!Y = GF(7)^4!},\quad{\ex\verb!T = Y.subspaces(2)!}\\
%\skipin\verb!T! is a generator object for 2-D subspaces of \verb!Y!\\
%\skipin\verb![U for U in T]! is list of 2850 2-D subspaces of \verb!Y!
{\ex\verb!V.dimension()!} \verb!V!'nin boyutu\\
{\ex\verb!V.basis()!} \verb!V!'nin tabanı\\
{\ex\verb!V.echelonized_basis()!}\\
{\ex\verb!V.has_user_basis()!} kullanıcı  \verb!V! için bir taban vermiş mi?\\
{\ex\verb!V.is_subspace(W)!} \verb!W!, \verb!V!'nin altuzayı mı?\\
{\ex\verb!V.is_full()!} modül olarak, rank mertebeye eşit mi?\\
{\ex\verb!Y = GF(7)^4!},\quad{\ex\verb!T = Y.subspaces(2)!}\\
\skipin\verb!T!, \verb!Y!'nin 2 boyutlu altuzaylarını üreten nesne\\
\skipin\verb![U for U in T]! \verb!Y!'nin 2 boyutlu altuzay listesi (=2850)
\par
%*********************************************
%\sect{Constructing Subspaces}
%{\ex\verb!span([v1,v2,v3], QQ)!} span of list of vectors over ring
%\par\vspace*{4pt}
%For a matrix \verb!A!, objects returned are\\
%\skipin vector spaces when base ring is a field\\
%\skipin modules when base ring is just a ring\\
%{\ex\verb!A.left_kernel() == A.kernel()!} \verb!right_! too\\
%{\ex\verb!A.row_space() == A.row_module()!}\\
%{\ex\verb!A.column_space() == A.column_module()!}\\
%{\ex\verb!A.eigenspaces_right()!} vectors on right, \verb!_left! too\\
%\skipin Pairs, having eigenvalue with its right eigenspace
%\par\vspace*{4pt}
%If \verb!V! and \verb!W! are subspaces\\
%{\ex\verb!V.quotient(W)!} quotient of \verb!V! by subspace \verb!W!\\
%{\ex\verb!V.intersection(W)!} intersection of \verb!V! and \verb!W!\\
%{\ex\verb!V.direct_sum(W)!} direct sum of \verb!V! and \verb!W!\\
%{\ex\verb!V.subspace([v1,v2,v3])!} specify basis vectors in a list
\sect{Altuzaylar}
{\ex\verb!span([v1,v2,v3], QQ)!} \verb!v1,v2,v3!'ün  gerdiği altuzay$/$\verb!QQ!
\par\vspace*{4pt}
Bir \verb!A! matrisinden başlayarak, \\
\skipin tanım halkası cisim ise, doğrusal uzaylar\\
\skipin tanım halkası yalnızca halka ise, modüller üretilir\\
{\ex\verb!A.left_kernel() == A.kernel()!} sağ yan için \verb!right_!\\
{\ex\verb!A.row_space() == A.row_module()!}\\
{\ex\verb!A.column_space() == A.column_module()!}\\
{\ex\verb!A.eigenspaces_right()!} vektörler sağda, sol için \verb!_left!\\
\skipin Pairs, having eigenvalue with its right eigenspace
\par\vspace*{4pt}
Eğer \verb!V! ve \verb!W! altuzaylarsa,\\
{\ex\verb!V.quotient(W)!} $= \verb!V!/  \verb!W! =$ \verb!V!'nin \verb!W! altuzayı ile bölümü\\
{\ex\verb!V.intersection(W)!} $= \verb!V! \cap \verb!W! =$ \verb!V!'nin \verb!W! ile kesişimi\\
{\ex\verb!V.direct_sum(W)!} $= \verb!V! \oplus \verb!W! =$ \verb!V! ile \verb!W!'nın direkt toplamı\\
{\ex\verb!V.subspace([v1,v2,v3])!} %% BURAYA
\par
%*********************************************
%\sect{Dense versus Sparse}
%{\warn Note:} Algorithms may depend on representation\\
%Vectors and matrices have two representations\\
%\skipin Dense: lists, and lists of lists\\
%\skipin Sparse: Python dictionaries\\
%{\ex\verb!.is_dense()!}, {\ex\verb!.is_sparse()!}  to check\\
%{\ex\verb!A.sparse_matrix()!} returns sparse version of \verb!A!\\
%{\ex\verb!A.dense_rows()!} returns dense row vectors of \verb!A!\\
%Some commands have  boolean \verb!sparse! keyword
\sect{Yoğun ya da Seyrek}
{\warn Not:} Algoritmalar, nesnelerin temsil şekline göre değişir\\
Vektörler ve matrislerin iki tür temsili vardır:\\
\skipin Yoğun: listeler (vektör) ve listelerin listeleri (matris)\\
\skipin Seyrek: Python sözlükleri\\
{\ex\verb!.is_dense()!} yoğun mu?, {\ex\verb!.is_sparse()!} seyrek mi?\\
{\ex\verb!A.sparse_matrix()!} \verb!A!'nın seyrek temsili\\
{\ex\verb!A.dense_rows()!} (yoğun) satır vektörleri olarak \verb!A!\\
Bazı komutlarda \verb!sparse! seçeneğini seçebilirsiniz 
\par
%*********************************************
%\sect{Rings}
%{\warn Note:} Many algorithms depend on the base ring\\
%{\ex\verb!<object>.base_ring(R)!} for vectors, matrices,\dots\\
%\skipin to determine the ring in use\\
%{\ex\verb!<object>.change_ring(R)!} for vectors, matrices,\dots\\
%\skipin to change to the ring (or field), \verb!R!,\\
%{\ex\verb!R.is_ring()!},\quad{\ex\verb!R.is_field()!}\\
%{\ex\verb!R.is_integral_domain()!},\quad{\ex\verb!R.is_exact()!}
%\par\vspace*{4pt}
%Some ring and fields\\
%\skipin{\ex\verb!ZZ!}\quad integers, ring\\
%\skipin{\ex\verb!QQ!}\quad rationals, field\\
%\skipin{\ex\verb!QQbar!}\quad algebraic field, exact\\
%\skipin{\ex\verb!RDF!}\quad real double field, inexact\\
%\skipin{\ex\verb!RR!}\quad 53-bit reals, inexact\\
%\skipin{\ex\verb!RealField(400)!}\quad 400-bit reals, inexact\\
%\skipin{\ex\verb!CDF!,\quad\verb!CC!,\quad\verb!ComplexField(400)!}\quad complexes, too\\
%\skipin{\ex\verb!RIF!}\quad real interval field\\
%\skipin{\ex\verb!GF(2)!}\quad mod 2, field, specialized implementations\\
%\skipin{\ex\verb!GF(p) == FiniteField(p)!}\quad \verb!p! prime, field\\
%\skipin{\ex\verb!Integers(6)!}\quad integers mod 6, ring only \\
%\skipin{\ex\verb!CyclotomicField(7)!}\quad rationals with 7$^{\rm th}$ root of unity\\
%\skipin{\ex\verb!QuadraticField(-5, 'x')!}\quad rationals adjoin \verb!x=!$\sqrt{-5}$\\
%\skipin{\ex\verb!SR!}\quad ring of symbolic expressions
\sect{Halkalar}
{\warn Not:} Farklı halkalar üzerinde farklı algoritmalar kullanılır\\
{\ex\verb!<nesne>.base_ring(R)!} vektör, matris gibi bir \verb!nesne! \\ 
\skipin için üzerinde tanımlandığı halkayı verir\\
{\ex\verb!<nesne>.change_ring(R)!} vektör, matris gibi bir \verb!nesne! 
\skipin için üzerinde tanımlandığı halkayı (cismi) değiştirir\\
{\ex\verb!R.is_ring()!} halka mı? ,\quad{\ex\verb!R.is_field()!}  cisim mi?\\
{\ex\verb!R.is_integral_domain()!},\quad{\ex\verb!R.is_exact()!}
\par\vspace*{4pt}
Bazı Halkalar ve Cisimler\\
\skipin{\ex\verb!ZZ!}\quad tamsayılar, halka\\
\skipin{\ex\verb!QQ!}\quad kesir sayılar, cisim\\
\skipin{\ex\verb!QQbar!}\quad cebirsel cisim, kayıpsız\\
\skipin{\ex\verb!RDF!}\quad real double field, kayıpsız\\
\skipin{\ex\verb!RR!}\quad 53-bit hassas gerçeller, kayıplı\\
\skipin{\ex\verb!RealField(400)!}\quad 400-bit hassas gerçeller, kayıpsız\\
\skipin{\ex\verb!CDF!,\quad\verb!CC!,\quad\verb!ComplexField(400)!}\quad karmaşıklar\\
\skipin{\ex\verb!RIF!}\quad gerçel aralık cismi\\
\skipin{\ex\verb!GF(2)!}\quad mod 2, cisim, özel algoritmalar\\
\skipin{\ex\verb!GF(p) == FiniteField(p)!}\quad \verb!p! asal, sonlu cisim\\
\skipin{\ex\verb!Integers(6)!}\quad mod 6'da tamsayılar, yalnızca halka \\
\skipin{\ex\verb!CyclotomicField(7)!}\quad 1'in 7. kökleri eklenmiş kesirler \\
\skipin{\ex\verb!QuadraticField(-5, 'x')!}\, kesirlere \verb!x=!$\sqrt{-5}$ eklenmiş\\
\skipin{\ex\verb!SR!}\quad simgesel ifadeler halkası
\par
%*********************************************
%\sect{Vector Spaces versus Modules}
%A module is ``like'' a vector space over a ring, not a field\\
%Many commands above apply to modules\\
%Some ``vectors'' are really module elements
\sect{Doğrusal Uzaylarla Modüller}
Bir halka üzerinde tanımlanan doğrusal uzaya modül denir \\
Yukarıdaki pek çok komut modüller için de geçerlidir\\
Bazı ``vektörler'' de aslında bir modülün elemanlarıdır
\par
%*********************************************
%\sect{More Help}
%``tab-completion'' on partial commands\\
%``tab-completion'' on \verb!<object.>! for all relevant methods\\
%\verb!<command>?! for summary and examples\\
%\verb!<command>??! for complete source code
\sect{Biraz Daha Yardım}
Kısmen yazılmış bir komut ``tab''a basarak tamamlanır\\
\verb!nesne.! 'den sonra ``tab'', ilgili tüm yöntemleri gösterir\\
\verb!<komut>?! özet ve örnekler\\
\verb!<komut>??! tüm kaynak kodu
%
\end{multicols*}

\end{document}
